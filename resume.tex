\documentclass[a4paper,10pt]{article}
\usepackage{graphicx}
\usepackage{float}
\usepackage{multirow}
\usepackage[left=0.75in,right=0.75in,top=0.75in,bottom=0.75in]{geometry}
\usepackage{fontspec}
\usepackage{textcomp}
\usepackage{setspace}
\newcommand{\header}[1]{
\begin{center}
\fbox{\begin{minipage}{\textwidth}
\raggedright \large \bf #1
\end{minipage}}
\end{center}
\vspace{-0.3cm}
}
\pagestyle{empty}
\begin{document}
\setmainfont{Ubuntu}
\begingroup
%\setlength{\tabcolsep}{0.05\textwidth}

% \begin{table}[h!]
%  \centering
%  \begin{tabular}{l l l}
%  \hspace{-1cm}\multirow{4}{0.2\textwidth}{\includegraphics{images/logo}} & \bf Sushant Mahajan & \bf 133059007 \\
%  & \bf Computer Science \& Engineering & \bf M.Tech. \\
%  & \bf Indian Institute of Technology, Bombay & \bf Male \\
%  & & \bf DOB: 28 Nov 1988 \\  
%  \end{tabular}
% \end{table}
% \endgroup
% 
% \begin{table}[h!]
%   \centering
%   \begin{tabular}{l l l l l}
%   \hline
%   \bf Examination & \bf University & \bf Institute & \bf Year & \bf CPI/\% \\
%   \hline
%   Post Graduation & IIT Bombay & IIT Bombay & 2016 & 9.08 \\
%   Undergraduate Specialization: CSE & & & & \\
%   Graduation & JIIT(Deemed) & JIIT, Noida & 2011 & 7.7 \\
%   Intermediate/+2 & CISCE & St. Joseph's Academy, Dehradun & 2006 & 92.5\% \\
%   Matriculation & CISCE & St. Joseph's Academy, Dehradun & 2004 & 89\% \\
%   \hline
%   \end{tabular}
% \end{table}

\vspace*{2.6in}
\vspace{-1.5cm}
\header{Awards and Achievements}
\vspace{-0.2cm}
\begin{itemize}
 \item Secured percentile of \textbf{99.81 amongst 224160} students in \textbf{GATE 2013}.
 \item \textbf{Oracle\textregistered ~Certified Professional} JAVA SE 6 Programmer. Cleared OCP-JP 6, 2014 with 90\%.
 \item \textbf{Microsoft\textregistered ~Specialist} in programming in HTML5 with JavaScript and CSS3, 2012. Cleared the certification with 76\%.
 \item Secured rank of \textbf{6 in 200} students in college's annual computer science conference \textbf{IC3, 2010} for the project \textbf{Voice controlled obstacle detector}.
 \item \textbf{NIIT - Sun certificate} for excellent performance in \textbf{JAVA 6(core)} course, 2009. 
 \item \textbf{1\textsuperscript{st} in Debugging/C++} programming contest in college's annual fest \textbf{JIVE, 2009}.
 \item Participated in \textbf{ACM} Kanpur, and received certificate of achievement. 
\end{itemize}

\vspace{-0.8cm}
\header{Fields of Interest}
\vspace{0.2cm}
Cryptography, Cloud computing, Software Architecture

\vspace{-0.4cm}
\header{Industry Experience - 24 months}
\vspace{0.2cm}
\noindent{Worked on a project in Healthcare domain for a Denmark based client, in a team of 13. The software is used by clinicians at hospitals in Denmark, where healthcare is mostly public. The main use case is to document and co-ordinate all activities related to treatment of a patient. I worked on 3 external and 1 internal modules.
\begin{itemize}
 \item \textbf{Module 1:} Enables \textbf{creation, saving and editing of RTF text} in admin application and provide the API for the usage of the text in the main client module. \\
 \textbf{Responsibilities:} Technical design, implementation, testing.
 
 \item \textbf{Module 2:} Did work relating to extendability of the module by changing the core design patterns. \\
 \textbf{Responsibilities:} Technical design, POC for design patterns, implementation, testing.
 
 \item \textbf{Module 3:} Extended list based design to include working with multiple items. Extension of module 2. \\
 \textbf{Responsibilities:} Implementation, testing.
 
 \item \textbf{Module 4-Internal:} Aimed at reducing the time required for preparing a new Environment to run the application and JUnit test cases on it, from a few days to a few hours. It helps a developer/tester to load appropriate data required into the DB. This data includes terminology codes, constants, ids, names, profile options and even XML template files. It can be compared to a class loader in JAVA. \\
 \textbf{Responsibilities:} Create user stories, Analysis of codes, implementation, review, documentation, testing.
\end{itemize}
\textbf{Technologies/Tools:} JAVA SE-6(Core), Swing, XML, Oracle, HTB DB Framework, Design Patterns, SONAR Quality tool, UISpec4J UI testing tool, QAPlug source code quality, SVN.
}

\vspace{-0.4cm}
\header{MTech Courses}
\vspace{0.2cm}
\begin{center}
$\bullet$Program Analysis\hspace*{0.3cm}$\bullet$Network Security and Cryptography II\hspace*{0.3cm}$\bullet$Natural Language Processing\\$\bullet$Software  Architecture\hspace*{0.3cm}$\bullet$Computer Networks\hspace*{0.3cm}$\bullet$Design and Implementation of GCC Framework\\
$\bullet$Engineering a Cloud\hspace*{0.3cm}$\bullet$An Introduction to Number Theory and Cryptography\newpage
\end{center}

\vspace{-0.4cm}
\header{Post Graduate Research/Projects}
\vspace{0.2cm}
\begingroup
\linespread{0.5}
\textbf{Homomorphic Encryption with application to SCM -} \emph{M.Tech. Thesis}\hfill[Autumn 2015 - present] \\
\emph{Guide: Prof. Bernard Menezes}
\begin{itemize}
 \item Goal: Implement a solution which enables PKE based evaluation of functions on encrypted data(integer vectors) such that decryption of the result is equivalent to applying the function on the original data. This has wide applications in machine learning and image processing. Our application is in the field of supply chain management - to predict the inventory needs based on encrypted past data.
\end{itemize}

\noindent
\textbf{Homomorphic Cryptography -} \emph{M.Tech. Seminar}\hfill[Spring 2014] \\
\emph{Guide: Prof. Bernard Menezes}
\begin{itemize}
 \item Goal: Survey various techniques that allow homomorphic encryption of data fully or partially. The relative efficacy of the techniques and the practicality of the implementation was also researched.
\end{itemize}

\noindent
\textbf{Cloud based memcache clone -} \emph{Course Project}\hfill[Spring 2014] \\
\emph{Guide: Prof. Sriram Srinivasan}
\begin{itemize}
 \item Goal: The memcache supports basic atomic instructions like set, get, cas for key value pairs. 5 servers maintain a replicated memcache table. The servers are cemented together with a RAFT based consensus algorithm.
 \item Technologies/Tools: Google go, git, lite IDE
\end{itemize}

\noindent
\textbf{Emotion detection from live chat -} \emph{Course Project}\hfill[Autumn 2014] \\
\emph{Guide: Prof. Pushpak Bhattacharyya}
\begin{itemize}
 \item Goal: Develop a system to analyze data from a text based chat, and as an output display the predicted emotion portrayed. We developed 3 models comprising chained application of prediction algorithms - naive bayes, linear svc and Multinomial Naive Bayes with TF-IDF and Chi$^2$ selection; Vector space models; frequency based prediction and analyzed the results. We also built the chat server-client application.
 \item Technologies/Tools: Python3, Tkinter, bash, nltk, wordnet
\end{itemize}

\noindent
\textbf{QT based SpecRTL  visualizer -} \emph{Course Project}\hfill[Spring 2013] \\
\emph{Guide: Prof. Uday Khedkar}
\begin{itemize}
 \item Goal: SpecRTL is a language for machine descriptions. Although easy to write it is very difficult to visualize the structures. We wrote a Qt based tool which takes as input the specRTL code and produces a tree based graphic output.
 \item Technologies/Tools: C/C++, Qt developer kit, specRTL
\end{itemize}
\endgroup

\vspace{-0.8cm}
\header{RA Work - System Administrator}
\vspace{-0.2cm}
\begin{itemize}
 \item Wrote scripts for automatic software installation, iptables manipulation and regular lab maintenance.
 \item Established the FOG server for automated creation and deployment of OS images on all lab systems, greatly reducing the man hours required.
 \item Deployed printer bank and wrote the web interface for printing documents using the department printers.
 \item Wrote scripts and programs for creating a printer accounting solution.
 \item Deployed iSCSI based filesystem that is used by all lab machines via NFS.
 \item Configured department servers with IPMI for low level management and emergency maintenance.
\end{itemize}

\vspace{-0.8cm}
\header{Skills}
\vspace{0.1cm}
\textbf{Languages} \hfill C/C++, JAVA 6(Core), Google Go, Python, bash \\
\textbf{OS} \hfill GNU/Linux (Ubuntu, Debian) \\
\textbf{Web technologies}\hfill HTML5, CSS3, JavaScript, PHP, JSP, Servlets, XML, JSON \\
\textbf{Tools} \hfill gnuplot, \LaTeX, git \\
\textbf{Database} \hfill MySql \\

\end{document}
